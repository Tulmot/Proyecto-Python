\capitulo{4}{Técnicas y herramientas}

En este apartado de la memoria se presentan las técnicas metodológicas y las herramientas de desarrollo que se han utilizado para llevar a cabo el proyecto. 

%\maketitle
\section{GitHub}
Es una plataforma para alojar proyectos y utiliza git como sistema de control de versiones. Se organiza por tareas (milestones e issues).
Utiliza el framework Ruby on Rails~\cite{github}.

Podemos acceder a él a través del siguiente enlace: 
\url{https://github.com/}

\textbf{Ventajas:}

\begin{itemize}
\item Es uno de los repositorios mas usados, por lo que es fácil encontrar información en Internet para resolver cualquier duda.

\item El código es público por lo que cualquiera puede proponer cambios en el mismo, seguirte y ver el proyecto.

\item Las distintas versiones del código están alojadas en la nube por lo que si perdemos el contenido de nuestro ordenador, podremos recuperarlo.
\end{itemize}

\textbf{Desventajas:}

\begin{itemize}
\item También puedes tener proyectos privados pero para ello tienes que utilizar una cuenta de pago, aunque los estudiantes e investigadores pueden obtener esto gratuitamente.
\end{itemize}


\section{Spyder}
Es un entorno de desarrollo interactivo para el lenguaje de Python, es de código abierto.
Tiene funciones avanzadas de edición, pruebas interactivas, depuración e introspección. También es un entorno informático numérico y tiene diversas bibliotecas que podemos utilizar, como pueden ser numpy ~\cite{spyder}.

Podemos acceder a él a través del siguiente enlace: 
\url{http://pythonhosted.org/spyder/}

\section{\LaTeX}
Se usa para la creación de documentos que necesiten una alta calidad tipográfica, como puede ser en artículos o libros científicos~\cite{latex}.

\textbf{Ventajas:}
\begin{itemize}
\item Es software libre, por lo que no requiere ningún coste.
\item No te tienes preocupar por el diseño, ya que la herramienta se encarga de ello.
\end{itemize}

\textbf{Desventajas:}
\begin{itemize}
\item Si eres principiante necesitas un tiempo de aprendizaje para saber como funciona.
\end{itemize}

Podemos acceder a él a través del siguiente enlace: 
\url{https://www.latex-project.org/}

\section{Jupyter Notebook}
Es una aplicación web de código abierto, con él podemos crear y compartir documentos, que nos permiten visualizar los resultados al ejecutar nuestro código, ya sean imágenes, árboles...que otros entornos de desarrollo (como Spyder mencionado anteriormente) no nos permiten esto.
Soporta más lenguajes, pero nosotros lo usaremos para el lenguaje de Python ~\cite{jupyter}.

Podemos acceder a él a través del siguiente enlace: 
\url{http://jupyter.org/}	

\section{Scikit-learn}
Es una librería de aprendizaje de software libre del lenguaje de programación de Python. Es una herramienta de las más utilizadas para la minería de datos y el análisis de datos~\cite{wiki:scikitlearn}. 
Esta basada en el aprendizaje automático, para ello se consideran un conjunto de n muestras y se intenta predecir las propiedades de los datos desconocidos. Podemos separar los problemas de aprendizaje en dos~\cite{scikitlearn}:

	\begin{itemize}
		\item Aprendizaje supervisado: En el que los datos vienen con atributos adicionales que queremos predecir. Dos de las tareas mas comunes son la clasificación y la regresión. En las de clasificación el programa debe aprender a predecir en que categoría o clase irán los nuevos datos según las nuevas observaciones, como sería predecir si el precio de una acción bajara o subirá. En lo de regresión el programa debe predecir el valor de una variable de respuesta continua, como sería predecir las ventas de un nuevo producto. 

		\item Aprendizaje no supervisado: Consiste en agrupar observaciones relaciones,dentro de los datos del entrenamiento. Clustering es la que más se utiliza para explorar un conjunto de datos.

	\end{itemize}
