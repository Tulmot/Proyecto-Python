\capitulo{3}{Conceptos teóricos}

\begin{itemize}
\item Scikit-learn: Es una librería de aprendizaje de software libre en el lenguaje de programación de Python. Es una herramienta de las más utilizadas para la minería de datos y el análisis de datos. 
Esta basada en el aprendizaje automático, para ello se consideran un conjunto de n muestras y se intenta predecir las propiedades de los datos desconocidos. Podemos separar los problemas de aprendizaje en dos:

\item Minería de Datos: Es un conjunto de reglas mediante las cuales se analizan gran des volúmenes de datos, la finalidad de esto es descubrir unos patrones, una similitud o una propensión que expliquen el comportamiento de los datos.
Para hacer esto utiliza los métodos de la inteligencia artificial, estadística y redes neuronales.
El objetivo del proceso de minería de datos consiste en extraer información de un conjunto de datos, luego se interpreta esta información para un uso posterior.
En general el proceso de la minería de datos consta de 4 etapas:
	\begin{itemize}
		\item Determinación de los objetivos: Se tratan los objetivos que quiere conseguir el cliente bajo un asesor especialista en minería de datos.
		\item Preprocesamiento de los datos: Es la etapa que más tiempo se tarda en realizar el el proceso. Se seleccionan, limpian, enriquecen, reducen y transforman las bases de datos. 
		\item Determinación del modelo: Se lleva a cabo un estudio estadístico de los datos, más tarde se hace una visualización gráfica para una primera aproximación. Según los objetivos que se habían propuesto se pueden usar diferentes algoritmos de la Inteligencia Artificial.
		\item Análisis de los resultados: Se comprueban si los datos obtenidos tienen coherencia, después se comparan con los obtenidos en los estudios estadísticos y la visualización gráfica. El cliente es el que ve si los datos le aportan nuevo conocimiento que le permita considerar sus decisiones.
	\end{itemize}

\end{itemize}

-Aprendizaje supervisado: En el que los datos vienen con atributos adicionales que queremos predecir. Dos de las tareas mas comunes son la clasificación y la regresión. En las de clasificación el programa debe aprender a predecir en que categoría o clase irán los nuevos datos según las nuevas observaciones, como sería predecir si el precio de una acción bajara o subirá. En lo de regresión el programa debe predecir el valor de una variable de respuesta continua, como sería predecir las ventas de un nuevo producto. 

-Aprendizaje no supervisado: Consiste en agrupar observaciones relaciones,dentro de los datos del entrenamiento. Clustering es la que más se utiliza para explorar un conjunto de datos.

Algunos conceptos teóricos de \LaTeX \footnote{Créditos a los proyectos de Álvaro López Cantero: Configurador de Presupuestos y Roberto Izquierdo Amo: PLQuiz}.

\section{Secciones}

Las secciones se incluyen con el comando section.

\subsection{Subsecciones}

Además de secciones tenemos subsecciones.

\subsubsection{Subsubsecciones}

Y subsecciones. 


\section{Referencias}

Las referencias se incluyen en el texto usando cite \cite{wiki:latex}. Para citar webs, artículos o libros \cite{koza92}.


\section{Imágenes}

Se pueden incluir imágenes con los comandos standard de \LaTeX, pero esta plantilla dispone de comandos propios como por ejemplo el siguiente:

\imagen{escudoInfor}{Autómata para una expresión vacía}



\section{Listas de items}

Existen tres posibilidades:

\begin{itemize}
	\item primer item.
	\item segundo item.
\end{itemize}

\begin{enumerate}
	\item primer item.
	\item segundo item.
\end{enumerate}

\begin{description}
	\item[Primer item] más información sobre el primer item.
	\item[Segundo item] más información sobre el segundo item.
\end{description}
	
\begin{itemize}
\item 
\end{itemize}

\section{Tablas}

Igualmente se pueden usar los comandos específicos de \LaTeX o bien usar alguno de los comandos de la plantilla.

\tablaSmall{Herramientas y tecnologías utilizadas en cada parte del proyecto}{l c c c c}{herramientasportipodeuso}
{ \multicolumn{1}{l}{Herramientas} & App AngularJS & API REST & BD & Memoria \\}{ 
HTML5 & X & & &\\
CSS3 & X & & &\\
BOOTSTRAP & X & & &\\
JavaScript & X & & &\\
AngularJS & X & & &\\
Bower & X & & &\\
PHP & & X & &\\
Karma + Jasmine & X & & &\\
Slim framework & & X & &\\
Idiorm & & X & &\\
Composer & & X & &\\
JSON & X & X & &\\
PhpStorm & X & X & &\\
MySQL & & & X &\\
PhpMyAdmin & & & X &\\
Git + BitBucket & X & X & X & X\\
Mik\TeX{} & & & & X\\
\TeX{}Maker & & & & X\\
Astah & & & & X\\
Balsamiq Mockups & X & & &\\
VersionOne & X & X & X & X\\
} 
