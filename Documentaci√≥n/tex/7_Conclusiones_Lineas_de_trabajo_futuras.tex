\capitulo{7}{Conclusiones y Líneas de trabajo futuras}

En esta sección, explicaremos las conclusiones a las que hemos llegado después de realizar el proyecto. También mencionaremos partes que se pueden mejorar en un futuro por desarrolladores. 

\section{Conclusiones}
Este apartado lo vamos a dividir en distintas partes, cada una con una conclusión. Con los aspectos más relevantes que hemos realizado.
\begin{itemize}
	\item Dinámica del proyecto: Estoy muy satisfecho con los aspectos que se han llevado a cabo. Algunas de las partes han supuesto un reto, por el desconocimiento del tema. Al conseguir completarlos han supuesto una satisfacción.
	\item Lenguaje de desarrollo: Se ha utilizado el lenguaje de Python, y hemos visto que para realizar cálculos, necesarios para nuestros algoritmos, dispone de una gran cantidad de librerías fáciles de usar, por ello para este proyecto Python es más eficaz que otro lenguaje.
	\item Aprendizaje: Este proyecto me ha servido para profundizar más en el aprendizaje de la minería de datos, y todo lo relacionado con la captura, administración y procesamiento de grandes cantidades de datos. En cuanto al lenguaje elegido, Python que mayores conocimientos he adquirido de la carrera, y por lo tanto he trabajado con mayor agilidad durante el desarrollo; me ayudado ha ampliar y afianzar los conocimientos de este lenguaje. 
\end{itemize}

\section{Lineas de trabajo futuras}
Este apartado vamos a mencionar diferentes mejoras para el proyecto.
\begin{itemize}
	\item Algunos de los métodos/funciones, se podrían simplificar su complejidad algorítmica, con el objetivo de agilzizarlos.
	\item Se podría implementar los algoritmos en otros lenguajes, que estén preparados para realizar cálculos, ya que sino se ralentizaría mucho el análisis de los datos. Uno de los lenguajes aconsejables sería \texttt{R} 
\end{itemize}