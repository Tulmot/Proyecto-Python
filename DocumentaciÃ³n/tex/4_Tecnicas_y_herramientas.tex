\capitulo{4}{Técnicas y herramientas}

En este apartado de la memoria se presentan las técnicas metodológicas y las herramientas de desarrollo que se han utilizado para llevar a cabo el proyecto. 

%\maketitle
\section{GitHub}
Es una plataforma para alojar proyectos y utiliza git como sistema de control de versiones. Se organiza por tareas (milestones e issues).
Utiliza el framework Ruby on Rails~\cite{github}.

Podemos acceder a través de: \url{https://github.com/}

\textbf{Ventajas:}

\begin{itemize}
\item Es uno de los repositorios mas usados, por lo que es fácil encontrar información en Internet para resolver cualquier duda.

\item El código es público por lo que cualquiera puede proponer cambios en el mismo, seguirte y ver el proyecto.

\item Las distintas versiones del código están alojadas en la nube por lo que si perdemos el contenido de nuestro ordenador, podremos recuperarlo.
\end{itemize}

\textbf{Desventajas:}

\begin{itemize}
\item También puedes tener proyectos privados pero para ello tienes que utilizar una cuenta de pago, aunque los estudiantes e investigadores pueden obtener esto gratuitamente.
\end{itemize}

\section{Python}
Python es un lenguaje de programación interpretado se hace hincapié en que una sintaxis que favorezca un código legible~\cite{python}.
Se trata de un lenguaje de programación multiparadigma (permite crear programas utilizando mas de un estilo de programación), ya que soporta orientación a objetos, programación imperativa y programación funcional. Es un lenguaje interpretado, usa tipado dinámico (una variable puede tomar valores de distinto tipo) y es multiplataforma~\cite{wiki:python}.

Python es recomendable para programadores que empiezan por primera vez, o que vienen de ostros lenguajes, ya que hay mucha documentación para dar el primer paso en este lenguaje. La comunidad organiza conferencias y también colabora con el código.

Contiene miles de módulos de terceros, a parte de la biblioteca estándar de Python, por lo que tenemos infinitas posibilidades.

Por último, Python se desarrolla bajo una licencia de código abierto aprobada por OSI, por lo que es se puede utilizar libremente y distribuir. La licencia de Python es administrada por la Python Software Foundation. 

Podemos acceder a través de: 
\url{https://www.python.org/}

\section{Spyder}
Es un entorno de desarrollo interactivo para el lenguaje de Python, es de código abierto.
Tiene funciones avanzadas de edición, pruebas interactivas, depuración e introspección. También es un entorno informático numérico y tiene diversas bibliotecas que podemos utilizar, como pueden ser numpy~\cite{spyder}.

Podemos acceder a través de: 
\url{http://pythonhosted.org/spyder/}

\section{\LaTeX}
Se usa para la creación de documentos que necesiten una alta calidad tipográfica, como puede ser en artículos o libros científicos~\cite{latex}.

Podemos acceder a través de: 
\url{https://www.latex-project.org/}

\textbf{Ventajas:}
\begin{itemize}
\item Es software libre, por lo que no requiere ningún coste.
\item No te tienes preocupar por el diseño, ya que la herramienta se encarga de ello.
\end{itemize}

\textbf{Desventajas:}
\begin{itemize}
\item Si eres principiante necesitas un tiempo de aprendizaje para saber como funciona.
\end{itemize}

\section{Jupyter Notebook}
Es una aplicación web de código abierto, con él podemos crear y compartir documentos, que nos permiten visualizar los resultados al ejecutar nuestro código, ya sean imágenes, árboles..., que otros entornos de desarrollo (como Spyder mencionado anteriormente) no nos permiten esto.
Soporta más lenguajes, pero nosotros lo usaremos para el lenguaje de Python~\cite{jupyter}.

Podemos acceder a través de: 
\url{http://jupyter.org/}	

\section{Scikit-learn}
Es una librería de Python que contiene algoritmos de aprendizaje automático para problemas supervisados y no supervisados. Como esta basado en Python, puede integrarse fácilmente en aplicaciones que no suelen usarse para análisis de datos estdísticos. EL trabajo futuro incluye aprendizaje en línea para escalar a grandes conjuntos de datos~\cite{scikitlearn}. 

Podemos acceder a través de: 
\url{http://scikit-learn.org/stable/}

\section{SonarQube}
Es una plataforma que sirve para evaluar y analizar código. Es software libre, para llevar a cabo este análisis del código utiliza distintas herramientas como pueden ser Checkstyle, PMD o FindBugs, con dichas herramientas obtenemos métricas que nos ayudan a mejorar la calidad de nuestro código fuente~\cite{wiki:sonarqube}.

Los aspectos que evalúa esta herramienta son:
\begin{itemize}
	\item Technical Debt, esta parte nos indica los aspectos y métricas que no habíamos tenido en cuenta y también nos muestran la claridad del código. Una de las ventajas es que te indica donde has cometido una falta de estilo o donde tienes demasiada complejidad.
	\item La complejidad, los cambios de flujo que sufre el código, es decir, las condiciones if, while, for… 
	\item Podemos ver las líneas de código que hemos escrito en cada fichero (sin contar los comentarios).
	\item Si queremos evaluar más aspectos, podemos instalar Plugins que nos lo permiten.
\end{itemize}

Podemos acceder a través de: 
\url{https://www.sonarqube.org/}

\section{Graphviz}
Es un software de visualización gráfica de código abierto. Es una forma de representar nuestra información estructural como diagramas de gráficos y redes abstractas. En nuestro caso lo utilizamos para dibujar un  árbol de nuestro conjunto de datos entrenado. Su arquitectura consiste en un lenguaje de descripción de gráficos llamado DOT.

Podemos acceder a través de: 
\url{https://www.graphviz.org/}

\section{Zenhub}
Es un gestor de tareas es similiar a Trello \url{https://trello.com/}, tiene un modo pizarra en el que podemos ver los cambios. Una de las grandes ventajas es que podemos integrarlo desde GitHub, por lo que no es necesario el uso de una aplicación externa. Una pequeña desventaja podríamos decir que es que no se puede añadir código, pero como el repositorio de GitHub nos permite visualizar dicho código no es un gran problema.

Podemos acceder a través de: 
\url{https://www.zenhub.com/}